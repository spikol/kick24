% Options for packages loaded elsewhere
\PassOptionsToPackage{unicode}{hyperref}
\PassOptionsToPackage{hyphens}{url}
%
\documentclass[
]{article}
\usepackage{amsmath,amssymb}
\usepackage{lmodern}
\usepackage{iftex}
\ifPDFTeX
  \usepackage[T1]{fontenc}
  \usepackage[utf8]{inputenc}
  \usepackage{textcomp} % provide euro and other symbols
\else % if luatex or xetex
  \usepackage{unicode-math}
  \defaultfontfeatures{Scale=MatchLowercase}
  \defaultfontfeatures[\rmfamily]{Ligatures=TeX,Scale=1}
\fi
% Use upquote if available, for straight quotes in verbatim environments
\IfFileExists{upquote.sty}{\usepackage{upquote}}{}
\IfFileExists{microtype.sty}{% use microtype if available
  \usepackage[]{microtype}
  \UseMicrotypeSet[protrusion]{basicmath} % disable protrusion for tt fonts
}{}
\makeatletter
\@ifundefined{KOMAClassName}{% if non-KOMA class
  \IfFileExists{parskip.sty}{%
    \usepackage{parskip}
  }{% else
    \setlength{\parindent}{0pt}
    \setlength{\parskip}{6pt plus 2pt minus 1pt}}
}{% if KOMA class
  \KOMAoptions{parskip=half}}
\makeatother
\usepackage{xcolor}
\IfFileExists{xurl.sty}{\usepackage{xurl}}{} % add URL line breaks if available
\IfFileExists{bookmark.sty}{\usepackage{bookmark}}{\usepackage{hyperref}}
\hypersetup{
  hidelinks,
  pdfcreator={LaTeX via pandoc}}
\urlstyle{same} % disable monospaced font for URLs
\setlength{\emergencystretch}{3em} % prevent overfull lines
\providecommand{\tightlist}{%
  \setlength{\itemsep}{0pt}\setlength{\parskip}{0pt}}
\setcounter{secnumdepth}{-\maxdimen} % remove section numbering
\ifLuaTeX
  \usepackage{selnolig}  % disable illegal ligatures
\fi

\author{}
\date{}

\begin{document}

The principles of variable scoping in Processing.py largely follow those
of Python but with some considerations given the environment of
Processing's draw loop and function setup.

\begin{enumerate}
\def\labelenumi{\arabic{enumi}.}
\item
  \textbf{Global Variables}: Variables declared outside of any function
  are global to the sketch. They can be accessed and modified from any
  function, but if you want to modify them inside a function, you need
  to declare them as \texttt{global} within that function.

\begin{verbatim}
pythonCopy code
x = 10

def changeX():
    global x
    x = 20
\end{verbatim}
\item
  \textbf{Local Variables}: Variables declared inside a function are
  local to that function. They cannot be accessed outside of the
  function, and their memory is reclaimed once the function execution is
  complete.

\begin{verbatim}
pythonCopy code
def showValue():
    y = 15
    print(y)  # This will print 15
    
showValue()
# print(y)  # This would be an error because y is not defined outside of the function.
\end{verbatim}
\item
  \textbf{The setup() and draw() Functions}: In Processing.py,
  \texttt{setup()} is called once at the beginning of the sketch, and
  \texttt{draw()} is called repeatedly, producing frames. Variables
  declared in \texttt{setup()} are local to
  \texttt{setup().\ Still,\ often\ you\textquotesingle{}ll\ want\ to\ declare\ global\ variables\ at\ the\ top\ level\ of\ your\ sketch\ and\ then\ initialize\ or\ modify\ them\ in}setup()`.

\begin{verbatim}
pythonCopy code
x = 0

def setup():
    global x
    size(400, 400)
    x = width / 2  # Initializing x based on the canvas width
    
def draw():
    global x
    background(240)
    ellipse(x, height/2, 50, 50)
    x += 1
\end{verbatim}
\end{enumerate}

\end{document}
