% Options for packages loaded elsewhere
\PassOptionsToPackage{unicode}{hyperref}
\PassOptionsToPackage{hyphens}{url}
%
\documentclass[
]{article}
\usepackage{amsmath,amssymb}
\usepackage{lmodern}
\usepackage{iftex}
\ifPDFTeX
  \usepackage[T1]{fontenc}
  \usepackage[utf8]{inputenc}
  \usepackage{textcomp} % provide euro and other symbols
\else % if luatex or xetex
  \usepackage{unicode-math}
  \defaultfontfeatures{Scale=MatchLowercase}
  \defaultfontfeatures[\rmfamily]{Ligatures=TeX,Scale=1}
\fi
% Use upquote if available, for straight quotes in verbatim environments
\IfFileExists{upquote.sty}{\usepackage{upquote}}{}
\IfFileExists{microtype.sty}{% use microtype if available
  \usepackage[]{microtype}
  \UseMicrotypeSet[protrusion]{basicmath} % disable protrusion for tt fonts
}{}
\makeatletter
\@ifundefined{KOMAClassName}{% if non-KOMA class
  \IfFileExists{parskip.sty}{%
    \usepackage{parskip}
  }{% else
    \setlength{\parindent}{0pt}
    \setlength{\parskip}{6pt plus 2pt minus 1pt}}
}{% if KOMA class
  \KOMAoptions{parskip=half}}
\makeatother
\usepackage{xcolor}
\IfFileExists{xurl.sty}{\usepackage{xurl}}{} % add URL line breaks if available
\IfFileExists{bookmark.sty}{\usepackage{bookmark}}{\usepackage{hyperref}}
\hypersetup{
  hidelinks,
  pdfcreator={LaTeX via pandoc}}
\urlstyle{same} % disable monospaced font for URLs
\setlength{\emergencystretch}{3em} % prevent overfull lines
\providecommand{\tightlist}{%
  \setlength{\itemsep}{0pt}\setlength{\parskip}{0pt}}
\setcounter{secnumdepth}{-\maxdimen} % remove section numbering
\ifLuaTeX
  \usepackage{selnolig}  % disable illegal ligatures
\fi

\author{}
\date{}

\begin{document}

\begin{enumerate}
\def\labelenumi{\arabic{enumi}.}
\item
  \textbf{Print Statements}:

  \begin{itemize}
  \item
    Sometimes, the simplest methods are the most effective. Use
    \texttt{print()} to display the values of variables, the flow of the
    program, or to check if a specific part of the code is being
    executed. This can quickly help you locate where things might be
    going wrong.
  \end{itemize}
\item
  \textbf{Error Messages}:

  \begin{itemize}
  \item
    Always read error messages in the Processing console. They can give
    you precise information on what went wrong and where. For instance,
    a \texttt{NullPointerException} might indicate you're trying to use
    an object that hasn't been initialized.
  \end{itemize}
\item
  \textbf{Commenting Out}:

  \begin{itemize}
  \item
    If you're unsure which part of your code is causing the problem, try
    commenting out sections of it to isolate the problematic area. You
    can gradually uncomment sections to narrow down the issue.
  \end{itemize}
\item
  \textbf{Visual Feedback}:

  \begin{itemize}
  \item
    Since Processing is a graphical environment, use visual feedback to
    your advantage. For example, change colors, draw borders, or use
    simple shapes to visually represent the flow of logic or the state
    of specific variables.
  \end{itemize}
\item
  \textbf{Break Down the Problem}:

  \begin{itemize}
  \item
    If you have a complex piece of code that isn't working, break it
    down into smaller, more manageable chunks. Test each chunk
    independently to ensure it works as expected before integrating it
    back into the larger codebase.
  \end{itemize}
\item
  \textbf{Check for Common Mistakes}:

  \begin{itemize}
  \item
    In the context of Processing, this might mean:

    \begin{itemize}
    \item
      Make sure \texttt{setup()} and \texttt{draw()} functions are
      correctly defined.
    \item
      Ensuring that you're using the right coordinates or dimensions for
      shapes.
    \item
      Verifying that image or data files are in the correct directory
      and are being loaded properly.
    \end{itemize}
  \end{itemize}
\end{enumerate}

\end{document}
