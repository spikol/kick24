% Options for packages loaded elsewhere
\PassOptionsToPackage{unicode}{hyperref}
\PassOptionsToPackage{hyphens}{url}
%
\documentclass[
]{article}
\usepackage{amsmath,amssymb}
\usepackage{lmodern}
\usepackage{iftex}
\ifPDFTeX
  \usepackage[T1]{fontenc}
  \usepackage[utf8]{inputenc}
  \usepackage{textcomp} % provide euro and other symbols
\else % if luatex or xetex
  \usepackage{unicode-math}
  \defaultfontfeatures{Scale=MatchLowercase}
  \defaultfontfeatures[\rmfamily]{Ligatures=TeX,Scale=1}
\fi
% Use upquote if available, for straight quotes in verbatim environments
\IfFileExists{upquote.sty}{\usepackage{upquote}}{}
\IfFileExists{microtype.sty}{% use microtype if available
  \usepackage[]{microtype}
  \UseMicrotypeSet[protrusion]{basicmath} % disable protrusion for tt fonts
}{}
\makeatletter
\@ifundefined{KOMAClassName}{% if non-KOMA class
  \IfFileExists{parskip.sty}{%
    \usepackage{parskip}
  }{% else
    \setlength{\parindent}{0pt}
    \setlength{\parskip}{6pt plus 2pt minus 1pt}}
}{% if KOMA class
  \KOMAoptions{parskip=half}}
\makeatother
\usepackage{xcolor}
\IfFileExists{xurl.sty}{\usepackage{xurl}}{} % add URL line breaks if available
\IfFileExists{bookmark.sty}{\usepackage{bookmark}}{\usepackage{hyperref}}
\hypersetup{
  hidelinks,
  pdfcreator={LaTeX via pandoc}}
\urlstyle{same} % disable monospaced font for URLs
\setlength{\emergencystretch}{3em} % prevent overfull lines
\providecommand{\tightlist}{%
  \setlength{\itemsep}{0pt}\setlength{\parskip}{0pt}}
\setcounter{secnumdepth}{-\maxdimen} % remove section numbering
\ifLuaTeX
  \usepackage{selnolig}  % disable illegal ligatures
\fi

\author{}
\date{}

\begin{document}

\begin{enumerate}
\def\labelenumi{\arabic{enumi}.}
\item
  \textbf{Basics of Processing IDE:}

  \begin{itemize}
  \item
    \textbf{Interface:} The Processing IDE has a simple interface.
    There's an area for writing code, buttons for running and stopping
    sketches, and a message area below for error messages and other
    notifications.
  \item
    \textbf{Sketch:} In Processing, each project is called a "sketch". A
    sketch is a combination of code, data, and output.
  \end{itemize}
\item
  \textbf{Language:} Processing uses a variant of the Java language.
  It's designed to be beginner-friendly, with simpler functions and
  setup to create visual and interactivey projects quickl.
\item
  \textbf{Structure of a Basic Sketch:}

  \begin{itemize}
  \item
    \textbf{\texttt{setup()} function:} This is executed once when the
    sketch starts. It's commonly used to define initial environment
    properties such as screen size and background color.
  \item
    \textbf{\texttt{draw()} function:} This runs repeatedly after
    \texttt{setup()}. It's used for continuously running code, such as
    animation or checking for input.
  \end{itemize}
\item
  \textbf{Running the Sketch:} The sketch is compiled and run when you
  click the "Run" button (or press Ctrl+R/Cmd+R). The visual output is
  displayed in a separate window.
\item
  \textbf{Libraries and Extensions:} Processing has a range of libraries
  that can be imported to add functionality, such as handling video,
  sound, or advanced graphics operations.
\item
  \textbf{Exporting:} You can export your sketches to standalone
  Windows, macOS, and Linux applications. Exporting projects for the web
  using the P5.js variant of Processing is also possible.
\item
  \textbf{Modes and Variants:}

  \begin{itemize}
  \item
    \textbf{P5.js:} A JavaScript library with the essence of Processing
    but for web development. It brings the Processing approach to web
    artists and developers.
  \item
    \textbf{Python Mode:} Allows you to write Processing sketches in
    Python.
  \item
    \textbf{Android Mode:} This lets you create Android apps using the
    Processing API.
  \end{itemize}
\item
  \textbf{Community:} One of the strengths of Processing is its active
  and supportive community. Numerous tutorials, forums, and resources
  are available to help users learn and troubleshoot.
\end{enumerate}

\end{document}
